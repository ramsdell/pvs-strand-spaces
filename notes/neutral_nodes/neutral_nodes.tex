\documentclass[12pt]{article}

% Ideas for making this into something more than just notes

% First algebra very simple to explain carried-by

% Add role orig assumptions later

% Add wrap-decrypt

\newif\ifreleased
\releasedfalse % \releasedtrue

% For drafts:
\ifreleased
\else
\pagestyle{myheadings}
\markright{Draft of \today{}}
\fi

\usepackage{graphics}
\usepackage[matrix,arrow,curve]{xy}
\usepackage{amssymb}
\usepackage{amsmath}
\usepackage{amsthm}
\usepackage{url}

\newtheorem{thm}{Theorem}
\newtheorem{lem}[thm]{Lemma}
\newcommand{\remark}[1]{\emph{[#1]}}
\newcommand{\cpsa}{\textsc{cpsa}}
\newcommand{\pvs}{\textsc{pvs}}
\newcommand{\cn}[1]{\ensuremath{\operatorname{\mathsf{#1}}}}
\newcommand{\fn}[1]{\ensuremath{\operatorname{\mathit{#1}}}}
\newcommand{\srt}[1]{\ensuremath{\mathsf{#1}}}
\newcommand{\typ}{\mathbin:}
\newcommand{\seq}[1]{\ensuremath{\langle#1\rangle}}
\newcommand{\enc}[2]{\ensuremath{\{\!|#1|\!\}_{#2}}}
\newcommand{\invk}[1]{{#1}^{-1}}
\newcommand{\tg}[1]{\cn{g}_{#1}}
\newcommand{\inbnd}{\mathord -}
\newcommand{\outbnd}{\mathord +}
\newcommand{\neutral}{\mathord\circ}
\newcommand{\nat}{\ensuremath{\mathbb{N}}}
\newcommand{\zed}{\ensuremath{\mathbb{Z}}}
%\newcommand{\pow}[1]{\wp(#1)}
\newcommand{\pow}[1]{\mathcal P(#1)}
\newcommand{\all}[1]{\forall#1\mathpunct.}
\newcommand{\some}[1]{\exists#1\mathpunct.}
\newcommand{\funct}[1]{\lambda#1\mathpunct.}
\newcommand{\up}{\mathord\uparrow}
\newcommand{\down}{\mathord\downarrow}

\newcommand{\nodes}{\fn{nd}}
\newcommand{\nnodes}{\operatorname{\mathit{nd}^{\neutral}}}
\newcommand{\key}{\srt{A}|\srt{S}}
\newcommand{\base}{\key|\srt{D}|\srt{E}}
\newcommand{\boot}{\cn{bt}}
\newcommand{\extend}{\cn{ex}}
\newcommand{\bootl}{\cn{bl}}
\newcommand{\extendl}{\cn{el}}
\newcommand{\observl}{\cn{ol}}
\newcommand{\tran}{\ensuremath{\tau}}
\newcommand{\pth}{\ensuremath{\pi}}
\newcommand{\pcr}{\fn{pcr}}
\newcommand{\evt}{\fn{evt}}
\newcommand{\sdom}{\fn{Dom}}
\newcommand{\ra}{\fn{ra}}

\title{Neutral Nodes in PVS}
\author{John D.~Ramsdell}

\begin{document}
\maketitle
%% \paragraph{Notation.}

%% A finite sequence is a function from an initial segment of the natural
%% numbers.  The length of a sequence~$X$ is~$|X|$, and
%% sequence~$X=\seq{X(0),\ldots, X(n-1)}$ for $n=|X|$.  If~$S$ is a set,
%% then~$S^\ast$ is the set of finite sequences over~$S$, and~$S^+$ is the
%% non-empty finite sequences over~$S$.  The prefix of sequence~$X$ of
%% length~$n$ is~$\prefix{X}{n}$.

\section{Message and State Model}
\label{sec:tpm}

Messages and states are modeled by elements of an order-sorted
algebra~\cite{GoguenMeseguer92}. An order-sorted algebra is a
generalization of a many-sorted algebra in which sorts may be
partially ordered.  The carrier sets associated with ordered sorts are
related by the subset relation.

\begin{figure}
$$\begin{array}{ll@{{}\typ{}}ll}
\mbox{Sorts:}&
\multicolumn{3}{l}{\mbox{\srt{M}, \srt{L}, $\top$, $\srt{A}$,
    $\srt{S}$, $\srt{D}$, $\srt{E}$}}\\
\mbox{Subsorts:}&
\multicolumn{3}{l}{\mbox{$\srt{A}<\top$, $\srt{S}<\top$,
    $\srt{D}<\top$, $\srt{E}<\top$}}\\
\mbox{Operations:}&\boot&\srt{M}&\mbox{Boot state}\\
&\extend&\top\times\srt{M}\to\srt{M}&\mbox{Extend state}\\
&\bootl&\srt{L}&\mbox{Boot label}\\
&\extendl&\srt{\top}\to\srt{L}&\mbox{Extend label}\\
&\observl&\srt{\top}\to\srt{L}&\mbox{Observe label}\\
&(\cdot,\cdot)&\top\times\top\to\top& \mbox{Pairing}\\
&\enc{\cdot}{(\cdot)}&\top\times\srt{A}\to\top&\mbox{Asymmetric encryption}\\
&\enc{\cdot}{(\cdot)}&\top\times\srt{S}\to\top&\mbox{Symmetric encryption}\\
&\#&\srt{\top}\to\top& \mbox{Hashing}\\
&\invk{(\cdot)}&\srt{A}\to\srt{A}& \mbox{Asymmetric key inverse}\\
&\invk{(\cdot)}&\srt{S}\to\srt{S}& \mbox{Symmetric key inverse}\\
&\cn{a}_i,\cn{b}_i&\srt{A}& \mbox{Asymmetric key constants}\\
&\cn{s}_i&\srt{S}& \mbox{Symmetric key constants}\\
&\cn{d}_i&\srt{D}& \mbox{Data constants}\\
&\cn{e}_i&\srt{E}& \mbox{Text constants}\\
&\tg{i}&\top& \mbox{Tag constants}\\
\mbox{Equations:}&\multicolumn{2}{l}{\invk{\cn{a}_i}=\cn{b}_i\quad
\invk{\cn{b}_i}=\cn{a}_i}
&(i\in\nat)\\
&\multicolumn{2}{l}{\all{k\typ\srt{A}}\invk{(\invk{k})}=k}
&\all{k\typ\srt{S}}\invk{k}=k
\end{array}$$
\caption{Crypto Algebra with State Signature}\label{fig:signature}
\end{figure}

Figure~\ref{fig:signature} shows the signature of the algebra used in
this paper.  Sort~\srt{M} is the sort of TPM machine states,
sort~\srt{L} is the sort of transition labels, and sort~$\top$ is the
sort of all messages.  Messages of sort~$\srt{A}$ (asymmetric keys),
sort~$\srt{S}$ (symmetric keys), sort~$\srt{D}$ (data), and
sort~$\srt{E}$ (text) are called \emph{atoms}.  Messages are atoms,
tag constants, or constructed using encryption $\enc{\cdot}{(\cdot)}$,
hashing $\#(\cdot)$, and pairing $(\cdot,\cdot)$, where the comma
operation is right associative and parentheses are omitted when the
context permits.  The canonical representative for each element in the
algebra is the term that contains the fewest number of occurrences of
the inverse operation~$\invk{(\cdot)}$.

The value in the PCR associated with a TPM state is the message
given by the {\pcr} function,
$$\begin{array}{r@{{}={}}l}
\multicolumn{2}{c}{\pcr\typ\srt{M}\to\top}\\
\pcr(\boot)&\cn{g}_0\\
\pcr(\extend(t, m))&\#(t,\pcr(m))
\end{array}$$
The function {\pcr} is injective.

The TPM transition relation
is~$\tran\subseteq\srt{M}\times\srt{L}\times\srt{M}$, where
$(m,\bootl,\boot)\in\tran$ (boot), $(m,\extendl(t),\extend(t,m))\in\tran$
(extend), or $(m,\observl(\pcr(m)), m)\in\tran$ (observe).  An infinite
sequence of states~$\pth_s$ and labels~$\pth_\ell$ is a \emph{path} if
$\all{i\in\nat}(\pth_s(i),\pth_\ell(i),\break\pth_s(i+1))\in\tran$
where $\pth_s(0)=\boot$.  Observation transitions are used with the
TPM quote and decrypt operations.

Useful properties about paths through state space are given by the Init
Extend and Prefix Extend Lemmas.

%% init_extend: lemma
%%   forall(p: path, x: mesg, st: state, k: nat):
%%     p(k)`1 = extend(x, st) implies
%%       exists(j: nat):
%%         j < k and p(j)`1 = st and p(j+1)`1 = extend(x, st)

\begin{lem}[Init Extend]\label{lem:init extend}
\begingroup\rm
$$\begin{array}{l}
\all{\pth_s,t\typ\top,m\typ\srt{M},k\in\nat}\\
\quad\mbox{$\pth_s(k)=\extend(t,m)$ implies}\\
\qquad\some{j\in\nat}
\mbox{$j<k$ and $\pth_s(j)=m$ and $\pth_s(j+1)=\extend(t,m)$}
\end{array}$$
\endgroup
\end{lem}

%% prefix_extend: lemma
%%   forall(p: path, x: mesg, st: state, i, k: nat):
%%     i <= k and p(k)`1 = extend(x, st) implies
%%       subterm(p(i)`1, p(k)`1) or
%%       exists(j: nat):
%%         i <= j and j < k and p(j)`1 = st and p(j+1)`1 = extend(x, st)

\begin{lem}[Prefix Extend]\label{lem:prefix extend}
\begingroup\rm
$$\begin{array}{l}
\all{\pth_s,t\typ\top,i,k\in\nat}\\
\quad\mbox{$i\leq k$ and $\pth_s(k)=\extend(t,m)$ implies}\\
\qquad\mbox{$\pth_s(i)$ is a subterm of $\pth_s(k)$ or}\\
\qquad\some{j\in\nat}
\mbox{$i\leq j<k$ and $\pth_s(j)=m$ and $\pth_s(j+1)=\extend(t,m)$}
\end{array}$$
\endgroup
\end{lem}

These two lemmas are used to prove the one employed to prove the
security goal in the envelope protocol.

%% state_split: lemma
%%   forall(p: path, x, x0, x1: mesg, st: state, i, k: nat):
%%     i <= k and x0 /= x1 and
%%     p(i)`1 = extend(x0, extend(x, st)) and
%%     p(k)`1 = extend(x1, extend(x, st)) implies
%%       exists(j0, j1: nat):
%%         j0 < i and i < j1 and j1 < k and
%%         p(j0)`1 = st and p(j0+1)`1 = extend(x, st) and
%%         p(j1)`1 = st and p(j1+1)`1 = extend(x, st)

\begin{lem}[State Split]\label{lem:state split}
\begingroup\rm
$$\begin{array}{l}
\all{\pth_s,t,t_0,t_1\typ\top,m\typ\srt{M},i,k\in\nat}\\
\quad\mbox{$i\leq k$ and $\pth_s(i)=\extend(t_0,\extend(t,m))$ and}\\
\quad\mbox{$t_0\neq t_1$ and $\pth_s(k)=\extend(t_1,\extend(t,m))$ implies}\\
\qquad\some{j_0,j_1\in\nat}\\
\qquad\quad\mbox{$j_0<i<j_1<k$ and $\pth_s(j_0)=\pth_s(j_1)=m$ and}\\
\qquad\quad\mbox{$\pth_s(j_0+1)=\pth_s(j_1+1)=\extend(t,m)$}
\end{array}$$
\endgroup
\end{lem}

\section{Strand Spaces With State Synchronization}\label{sec:strand spaces}

In strand space theory, the \emph{trace} of a strand is a linearly
ordered sequence of events $e_0\Rightarrow\cdots\Rightarrow e_{n-1}$,
and an \emph{event} is a message transmission $\outbnd t$ or a
reception $\inbnd t$, where~$t$ has sort~$\top$, or a state
synchronization $\neutral\ell$, where~$\ell$ has sort~$\srt{L}$.  A
\emph{strand space}~$\Theta$ is a map from a set of strands to a set
of traces.  We choose the set of strands to be a prefix of the natural
numbers, so a strand space is finite sequence of traces.

A node names an event in a strand space.  The set of \emph{nodes} of
strand space $\Theta$ is $\{(s,i)\mid s\in\sdom(\Theta), 0\leq i <
|\Theta(s)|\}$, and the event at a node is
$\evt_\Theta(s,i)=\Theta(s)(i)$.  A node is a \emph{neutral node}
in~$\Theta$ if the event at the node is a state synchronization.  The set
of nodes of~$\Theta$ is $\nodes(\Theta)$ and the set of neutral nodes
of~$\Theta$ is $\nnodes(\Theta)$.

A message~$t_0$ is \emph{carried by}~$t_1$, written $t_0\sqsubseteq
t_1$ if~$t_0$ can be extracted from a reception of~$t_1$, assuming
plaintext is extractable from encryptions.  In other
words,~$\sqsubseteq$ is the smallest reflexive, transitive relation
such that $t_0\sqsubseteq t_0$, $t_0\sqsubseteq (t_0, t_1)$,
$t_1\sqsubseteq (t_0, t_1)$, and $t_0\sqsubseteq\enc{t_0}{t_1}$.

A message \emph{originates} in trace~$c$ at index~$i$ if it is carried
by $c(i)$, $c(i)$ is a transmission, and it is not carried by any
messaging event earlier in the trace.  A message~$t$ is
\emph{non-originating} in a strand space~$\Theta$, written
$\fn{non}(\Theta,t)$, if it originates on no strand.  A message~$t$
\emph{uniquely originates} in a strand space~$\Theta$ at node~$n$,
written $\fn{uniq}(\Theta,t,n)$, if it originates in the trace of
exactly one strand~$s$ at index~$i$, and $n=(s,i)$.

The model of execution is a bundle.  The triple
$\Upsilon=(\Theta,\to,\leadsto)$ is a \emph{bundle} if it defines a
finite directed acyclic graph, where the vertices are the nodes of
$\Theta$, and an edge represents communication~($\to$), state
passing~($\leadsto$), or strand succession~($\Rightarrow$)
in~$\Theta$.  For communication, if $n_0\rightarrow n_1$, then there
is a message~$t$ such that~$\evt_\Theta(n_0)=\outbnd t$
and~$\evt_\Theta(n_1)=\inbnd t$.  For each reception node~$n_1$, there
is a unique transmission node~$n_0$ with $n_0\to n_1$.  For state
passing, if $n_0\leadsto n_1$, then~$n_0$ and~$n_1$ are neutral nodes,
and transitions at neutral nodes are compatible with a path through
state space.  The transitions are \emph{compatible} with a path if
there is a one-to-one correspondence between neutral nodes and an
initial segment of a path through state space, the state passing edges
respect path ordering, and neutral node labels map to path labels.
More formally, transitions at neutral nodes are state
compatible~\cite[Def.~11]{Guttman12} if there exists an $i\in\nat$,
$f\in\nnodes(\Theta)\to\zed_i$, and path~$\pth$ such that
\begin{enumerate}
\item\label{enum:bijection} $f$ is a bijection,
\item\label{enum:orderings} $\all{n_0,n_1\in\nnodes(\Theta)}
n_0\leadsto n_1\mbox{ iff }f(n_0)+1=f(n_1)$, and
\item\label{enum:labels} $\all{n\in\nnodes(\Theta)}
\evt_\Theta(n)=\pth_\ell(f(n))$
\end{enumerate}
State compatibility implies that neutral nodes are totally ordered.

State compatibility Property~\ref{enum:labels} asserts that the label
at a neutral node must agree with the label of some path through state
space.  It is this property that explains why a labeled transition
system is used instead of a simple state transition system.  The label
asserts that a subset of the set of transitions are bound to the event
at a neutral node.  However, had a subset of the transition relation
been bound to the event, all that could have been assert by
Property~\ref{enum:labels} is that the path associated with the event
is in the subset.

Each acyclic graph has a transitive irreflexive relation~$\prec$ on
its vertices.  The relation specifies the causal ordering of nodes in
a bundle.  A transitive irreflexive binary relation is also called a
strict order.  In a bundle, when~$\prec$ is restricted to neutral
nodes, it is can be shown that it is identical to~$\leadsto^+$, where
$\leadsto^+$ is the transitive closure of~$\leadsto$.  Furthermore,
compatibility Property~\ref{enum:orderings} can be used to derive the
following relation between node ordering and the function~$f$.
$$\all{n_0,n_1\in\nnodes(\Theta)}n_0\prec n_1\mbox{ iff }f(n_0)<f(n_1)$$

In the remainder of this section, the theory of strand spaces used in
the proofs has been simplified.  In the full theory, origination
assumptions can be inherited from roles.  See~\cite{Ramsdell13} for
all the gory details.

When a bundle is a run of a protocol, the behavior of each strand is
constrained by a role.  Adversarial strands are constrained by roles
as are non-adversarial strands.  A \emph{protocol} is a set of roles,
and a \emph{role} is a function from message variables to a trace.  A
trace~$c$ is an \emph{instance} of role~$r$ if~$c$ is a prefix of~$r$
applied to some messages.  For protocol~$P$, bundle
$\Upsilon=(\Theta,\to,\leadsto)$ is a \emph{run of protocol}~$P$ if
there exists a role assignment $\ra\in \sdom(\Theta)\to P$ such that
for all $s\in\sdom(\Theta)$, $\Theta(s)$ is an instance of~$\ra(s)$.
In what follows, we fix the protocol~$P$ and only consider bundles
that are runs of~$P$.

\begin{figure}
$$\begin{array}{r@{{}={}}l}
\fn{create}(t\typ\base)&\outbnd t\qquad\fn{tag}_i=\outbnd\tg{i}\\
\fn{pair}(t_0\typ\top, t_1\typ\top)&
\inbnd t_0\Rightarrow\inbnd t_1\Rightarrow\outbnd (t_0,t_1)\\
\fn{sep}(t_0\typ\top, t_1\typ\top)&
\inbnd (t_0, t_1)\Rightarrow\outbnd t_0\Rightarrow\outbnd t_1\\
\fn{enc}(t\typ\top, k\typ\srt{A}|\srt{S})&
\inbnd t\Rightarrow\inbnd k\Rightarrow\outbnd \enc{t}{k}\\
\fn{dec}(t\typ\top, k\typ\key)&
\inbnd \enc{t}{k}\Rightarrow\inbnd\invk{k}\Rightarrow\outbnd t
\end{array}$$
\caption{Adversary Roles}\label{fig:adversary}
\end{figure}

The roles that constrain adversarial behavior are in
Figure~\ref{fig:adversary}.  For the encryption related roles,
$k\typ\key$ asserts that~$k$ is either a symmetric or asymmetric
key.  For the create role, $t\typ\base$ asserts that~$t$ is an atom.

\section{Skeletons}\label{sec:skeletons}

In this paper, a skeleton will be specified using a sentence in
order-sorted logic.  The sorts are the message algebra sorts augmented
with a sort~\srt{Z} for strands and sort~\srt{N} for nodes.  The
atomic formula $\cn{htin}(z,h,c)$ asserts that strand~$z$ has a length
of at least~$h$, and its trace is a prefix of trace~$c$. The formula
$n_0\ll n_1$ asserts node~$n_0$ precedes node~$n_1$.  The formula
$\cn{non}(t)$ asserts that message~$t$ is non-originating, and
$\cn{uniq}(t,n)$ asserts that message~$t$ uniquely originates at
node~$n$.  Finally, the formula $\cn{sends}(n,t)$ asserts that the
event at node~$n$ is a transmission of message~$t$.  The roles of the
protocol serve as function symbols.

\begin{figure}
$$\begin{array}{l@{\quad}l}
\Upsilon,\alpha\models x=y&\mbox{iff $\alpha(x)=\alpha(y)$;}\\
(\Theta,\to,\leadsto),\alpha\models\cn{htin}(z,h,c)
&\mbox{iff $|\Theta(\alpha(z))|\geq \alpha(h)$ and}\\
&\mbox{\phantom{iff} $\Theta(\alpha(z))$ is a prefix of $\alpha(c)$;}\\
\Upsilon,\alpha\models n_0\ll n_1
&\mbox{iff $\alpha(n_0)\prec_\Upsilon\alpha(n_1)$;}\\
(\Theta,\to,\leadsto),\alpha\models\cn{non}(t)
&\mbox{iff $\fn{non}(\Theta,\alpha(t))$;}\\
(\Theta,\to,\leadsto),\alpha\models\cn{uniq}(t,n)
&\mbox{iff $\fn{uniq}(\Theta,\alpha(t),\alpha(n))$;}\\
(\Theta,\to,\leadsto),\alpha\models\cn{sends}(n,t)
&\mbox{iff $\evt_\Theta(\alpha(n))=\outbnd\alpha(t)$.}
\end{array}$$
\caption{Satisfaction}\label{fig:satisfaction}
\end{figure}

For bundle~$\Upsilon$, variable assignment~$\alpha$, and
formula~$\Phi$, satisfaction $\Upsilon,\alpha\models\Phi$ is defined
using the semantics for atomic formulas specified in
Figure~\ref{fig:satisfaction}.  A bundle~$\Upsilon$ is described by a
skeleton iff the skeleton's sentence~$\Phi$ is modeled by~$\Upsilon$,
written $\Upsilon\models\Phi$.

\bibliography{secureprotocols}
\bibliographystyle{plain}

\end{document}
