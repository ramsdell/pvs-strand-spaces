\documentclass[12pt]{article}

%\renewcommand{\pagename}{Draft of \today{}, Page}

\usepackage{graphics}
\usepackage[matrix,arrow,curve]{xy}
\usepackage{amssymb}
\usepackage{amsmath}
\usepackage{amsthm}
\usepackage{url}
%\usepackage{hyperref}\hypersetup{colorlinks=true,}

\newtheorem{thm}{Theorem}
\newcommand{\cpsa}{\textsc{cpsa}}
\newcommand{\pvs}{\textsc{pvs}}
\newcommand{\cn}[1]{\ensuremath{\operatorname{\mathsf{#1}}}}
\newcommand{\dom}[1]{\ensuremath{\operatorname{\mathbf{#1}}}}
\newcommand{\fn}[1]{\ensuremath{\operatorname{\mathit{#1}}}}
\newcommand{\sdom}{\fn{Dom}}
\newcommand{\sran}{\fn{Ran}}
\newcommand{\seq}[1]{\ensuremath{\langle#1\rangle}}
\newcommand{\enc}[2]{\ensuremath{\{\!|#1|\!\}_{#2}}}
\newcommand{\inv}[1]{{#1}^{-1}}
\newcommand{\inbnd}{\mathord -}
\newcommand{\outbnd}{\mathord +}
\newcommand{\srt}[1]{\ensuremath{\mathsf{#1}}}
\newcommand{\nat}{\ensuremath{\mathbb{N}}}
\newcommand{\all}[1]{\forall#1\mathpunct.}
\newcommand{\some}[1]{\exists#1\mathpunct.}
\newcommand{\pow}[1]{\wp(#1)}
\newcommand{\prefix}[2]{#1\mid#2}
\newcommand{\init}{\fn{init}}
\newcommand{\resp}{\fn{resp}}
\newcommand{\form}{\mathcal{K}}
\newcommand{\sent}{\mathcal{S}}
\newcommand{\lang}{\mathcal{L}}
\newcommand{\alg}[1]{\ensuremath{\mathfrak#1}}
\newcommand{\alga}{\alg{A}}
\newcommand{\tr}{\ensuremath{\mathfrak C}}
\newcommand{\rl}{\fn{rl}}
\newcommand{\skel}{\mathsf{k}}
\newcommand{\insta}{\mathsf{i}}
\newcommand{\nodes}{\fn{nodes}}
\newcommand{\evt}{\fn{evt}}
\newcommand{\role}{\mathsf{r}}
\newcommand{\orig}{\mathcal{O}}

\newcommand{\boot}{\cn{bt}}
\newcommand{\extend}{\cn{ex}(\cn{d},\boot)}
\newcommand{\tran}{\ensuremath{\tau}}
\newcommand{\pth}{\ensuremath{\pi}}
\newcommand{\type}{\ensuremath{\mathfrak T}}
\newcommand{\up}{\mathord\uparrow}
\newcommand{\down}{\mathord\downarrow}

\title{Notes About the \\ Simple TPM Attester Protocol}
\author{John D.\ Ramsdell}

\begin{document}
\maketitle

See the associated MITRE Technical Report (MTR) before looking at
these notes.

\begin{figure}
$$\begin{array}{ll@{{}\colon{}}ll}
\mbox{Sorts:}&
\multicolumn{3}{l}{\mbox{$\top$, $\srt{A}$, $\srt{S}$, $\srt{D}$,
 $\srt{E}$, \srt{M}}}\\
\mbox{Subsorts:}&
\multicolumn{3}{l}{\mbox{$\srt{A}<\top$, $\srt{S}<\top$,
    $\srt{D}<\top$, $\srt{E}<\top$}}\\
\mbox{Operations:}&(\cdot,\cdot)&\top\times\top\to\top& \mbox{Pairing}\\
&\enc{\cdot}{(\cdot)}&\top\times\srt{A}\to\top&\mbox{Asymmetric encryption}\\
&\enc{\cdot}{(\cdot)}&\top\times\srt{S}\to\top&\mbox{Symmetric encryption}\\
&\inv{(\cdot)}&\srt{A}\to\srt{A}& \mbox{Asymmetric key inverse}\\
&\inv{(\cdot)}&\srt{S}\to\srt{S}& \mbox{Symmetric key inverse}\\
&\#&\srt{\top}\to\srt{S}& \mbox{Hashing}\\
&\cn{a}_i,\cn{b}_i&\srt{A}& \mbox{Asymmetric key constants}\\
&\cn{s}_i&\srt{S}& \mbox{Symmetric key constants}\\
&\cn{d}_i&\srt{D}& \mbox{Data constants}\\
&\cn{e}_i&\srt{E}& \mbox{Text constants}\\
&\cn{g}_i&\top& \mbox{Tag constants}\\
&\boot&\srt{M}&\mbox{TPM boot}\\
&\cn{ex}&\top\times\srt{M}\to\srt{M}&\mbox{TPM extend}\\
\mbox{Equations:}&\multicolumn{2}{l}{\inv{\cn{a}_i}=\cn{b}_i\quad
\inv{\cn{b}_i}=\cn{a}_i}
&(i\in\nat)\\
&\multicolumn{2}{l}{\all{k\colon\srt{A}}\inv{(\inv{k})}=k}
&\all{k\colon\srt{S}}\inv{k}=k
\end{array}$$
\caption{Crypto Algebra with State Signature}\label{fig:signature}
\end{figure}

\begin{figure}
  \begin{center}
    \includegraphics{stap-0.mps}
  \end{center}
  \caption{STAP Message-Passing and State History}\label{fig:shape}
\end{figure}

The Simple TPM Attester Protocol (STAP) message algebra displayed in
Figure~\ref{fig:signature} extents the one in the MTR by adding
hashing and tags.  It also adds the sort~\srt{M} for the state of the
TPM, and two operations~\cn{bt} and~\cn{ex}, for boot and extend.
Thus a state is a term of sort~\srt{M}.

$$\begin{array}{ll@{{}\colon{}}ll}
\mbox{Sorts:}&\multicolumn{3}{l}{\srt{M}}\\
\mbox{Operations:}&\boot&\srt{M}& \mbox{Boot}\\
&\cn{ex}&\top\times\srt{M}\to\srt{M}&\mbox{PCR extension}
\end{array}$$

The \emph{transition relation} is~$\tran$, where $(m_0,m_1)\in\tran$
iff $m_1=\boot$ (boot), $\some{t\colon\top}m_1=\cn{ex}(t,m_0)$
(extend), or $m_0 = m_1$ (observe).  An infinite sequence~$\pth$ is a
\emph{path} if $\all{i\in\nat}(\pth(i),\pth(i+1))\in\tran$.

The encoding of TPM states as messages follows.
$$\begin{array}{r@{{}={}}l}
\multicolumn{2}{c}{\fn{pcr}\colon\srt{M}\to\srt{S}}\\
\fn{pcr}(\boot)&\cn{s}_0\\
\fn{pcr}(\cn{ex}(t, m))&\#(t,\fn{pcr}(m))
\end{array}$$

  %% stable_boot_extend: lemma
  %%   forall(p: path, x: mesg, i, k: nat):
  %%     i < k and boot?(p(i)) and p(k) = extend(x, boot) implies
  %%       exists(j: nat):
  %%         i <= j and j < k and p(j) = boot and
  %%         forall(l: nat):
  %%           j < l and l <= k implies
  %%             p(l) = extend(x, boot)

Theorem~\ref{thm:stable boot extend} in the state world is
imported into the strand space world as a bridge lemma.
\begin{thm}[Stable Boot Extend]\label{thm:stable boot extend}
$$\begin{array}{l}
\all{\pi\in\fn{path},t\colon\top,i,j\in\nat}\\
\quad i<j\wedge\pi(i)=\boot\wedge\pi(k)=\cn{ex}(t,\boot)\supset{}\\
\qquad\some{j\in\nat}\\
\qquad\quad i\leq j\wedge j<k\wedge\pi(j)=\boot\wedge{}\\
\qquad\quad\all{\ell\in\nat}
j<\ell\wedge l\leq k\supset\pi(\ell)=\cn{ex}(t,\boot)
\end{array}$$
\end{thm}

\emph{Much text has yet to written following this point\ldots}

\paragraph{Annotated STAP Roles.}

Some of the tags used in the protocol.
$$\begin{array}{r@{{}={}}ll}
\cn{st}&\cn{g_0}&\mbox{State}\\
\cn{cd}&\cn{g_1}&\mbox{Key Created}\\
\cn{de}&\cn{g_2}&\mbox{Decrypt}\\
\cn{d}&\cn{g_3}&\mbox{Desired PCR Value}
\end{array}$$

\paragraph{STAP Shape.}

The shape and its connection to state is in Figure~\ref{fig:shape}.

\end{document}
