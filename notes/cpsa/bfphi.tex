\documentclass[12pt]{article}

\newif\ifreleased
\releasedfalse % \releasedtrue

% For drafts:
\ifreleased
\else
\pagestyle{myheadings}
\markright{Draft of \today{}}
\fi

%\usepackage{graphics}
%\usepackage[matrix,arrow,curve]{xy}
\usepackage{amssymb}
\usepackage{amsmath}
\usepackage{amsthm}
\usepackage{url}

\newtheorem{thm}{Theorem}
\newtheorem{lem}[thm]{Lemma}
\newtheorem{conj}[thm]{Conjecture}
\newenvironment{note}{\begingroup\itshape}{\endgroup}
\newcommand{\remark}[1]{\emph{[#1]}}
\newcommand{\cpsa}{\textsc{cpsa}}
\newcommand{\pvs}{\textsc{pvs}}
\newcommand{\cn}[1]{\ensuremath{\operatorname{\mathsf{#1}}}}
\newcommand{\fn}[1]{\ensuremath{\operatorname{\mathit{#1}}}}
\newcommand{\srt}[1]{\ensuremath{\mathsf{#1}}}
\newcommand{\gtag}[1]{\ensuremath{\mathit{#1}}}
\newcommand{\typ}{\mathbin:}
\newcommand{\seq}[1]{\ensuremath{\langle#1\rangle}}
\newcommand{\append}{\cdot}
\newcommand{\enc}[2]{\ensuremath{\{\!|#1|\!\}_{#2}}}
\newcommand{\invk}[1]{{#1}^{-1}}
\newcommand{\tg}[1]{\cn{g}_{#1}}
\newcommand{\inbnd}{\mathord -}
\newcommand{\outbnd}{\mathord +}
\newcommand{\start}{\mathord\ast}
\newcommand{\sync}{\mathord !}
\newcommand{\obsv}{\mathord ?}
\newcommand{\neutral}{\mathord\circ}
\newcommand{\nat}{\ensuremath{\mathbb{N}}}
\newcommand{\zed}{\ensuremath{\mathbb{Z}}}
\newcommand{\all}[1]{\forall#1\mathpunct.}
\newcommand{\some}[1]{\exists#1\mathpunct.}
\newcommand{\funct}[1]{\lambda#1\mathpunct.}
\newcommand{\sel}{\downarrow}

\newcommand{\alg}{\ensuremath{\mathbb{A}}}
\newcommand{\msg}{\ensuremath{\mathcal{M}}}
\newcommand{\ssp}{\ensuremath{\mathcal{S}}}
\newcommand{\bun}{\ensuremath{\mathcal{B}}}
\newcommand{\sta}{\ensuremath{\mathcal{Q}}}
\newcommand{\strands}{\ensuremath{\mathcal{Z}}}
\newcommand{\nodes}{\ensuremath{\mathcal{N}}}
\newcommand{\nnodes}{\nodes^{\,!}}
\newcommand{\key}{\srt{A}|\srt{S}}
\newcommand{\base}{\key|\srt{D}|\srt{E}}
\newcommand{\boot}{\cn{bt}}
\newcommand{\extend}{\cn{ex}}
\newcommand{\init}{\ensuremath{\mathcal{I}}}
\newcommand{\tran}{\ensuremath{\mathcal{T}}}
\newcommand{\pth}{\ensuremath{\pi}}
\newcommand{\encode}{\fn{enc}}
\newcommand{\decode}{\fn{dec}}
\newcommand{\eran}{\ensuremath{\mathcal{R}}}
\newcommand{\evt}{\fn{evt}}
\newcommand{\dom}{\fn{Dom}}
\newcommand{\ran}{\fn{Ran}}
\newcommand{\ra}{\fn{ra}}

\newcommand{\comp}{\ensuremath{\mathcal{C}}}
\newcommand{\family}{\ensuremath{\mathcal{F}}}
\newcommand{\pos}{\ensuremath{\mathcal{P}}}
\newcommand{\lab}{\ensuremath{\mathcal{L}}}
\newcommand{\labs}{\ensuremath{\lambda}}
\newcommand{\cbar}[1]{\ensuremath{\overline{#1}}}
\newcommand{\cnnodes}{\bar\nodes^{\neutral}}

\newcommand{\bool}{\ensuremath{\mathbb{B}}}
\newcommand{\truth}{\ensuremath{\mathfrak{t}}}
\newcommand{\falsehood}{\ensuremath{\mathfrak{f}}}
\newcommand{\gua}{\ensuremath{\mathcal{G}}}
\newcommand{\addr}{\ensuremath{\mathcal{A}}}
\newcommand{\attr}{\ensuremath{\mathcal{C}}}
\newcommand{\ainit}{\ensuremath{\mathsf{init}}}
\newcommand{\awrap}{\ensuremath{\mathsf{wrap}}}
\newcommand{\adecrypt}{\ensuremath{\mathsf{decrypt}}}
\newcommand{\anno}{\fn{anno}}

\newcommand{\cpsacopying}{\begingroup
  \renewcommand{\thefootnote}{}\footnotetext{{\copyright} 2015 The
    MITRE Corporation.  Permission to copy without fee all or part of
    this material is granted provided that the copies are not made or
    distributed for direct commercial advantage, this copyright notice
    and the title of the publication and its date appear, and notice
    in given that copying is by permission of The MITRE
    Corporation.

    Approved for Public Release; Distribution Unlimited. Case Number
    14-2229.}\endgroup}

\title{The \bun,\family,$\phi$ Model}
\author{John D.~Ramsdell\and Joshua D.~Guttman\and Moses D.~Liskov\and
  Paul D.~Rowe}

\begin{document}
\maketitle
\cpsacopying

In this note, we describe a model patterned after the
\bun,\cbar\comp,$\phi$ model~\cite{Guttman12}.  An execution of a protocol
is described by a set strands.  In this model, each strand is a
sequence of events of five kinds, \emph{transmissions},
\emph{receptions}, \emph{initializations}, \emph{transitions}, and
\emph{observations}.  Each event has a message with the exception of a
transition, which has two.  A transition event provides
synchronization between protocol activity and state change, and an
observation provides a view into the current state.  These changes
lead naturally to a \bun,\family,$\phi$ model, which is motivated by the
recent implementation of state semantics in CPSA~3, and is a synthesis
of the various models proposed by the authors.

\emph{Much more should be here someday.}

In the text that follows, an exclamation point in the margin marks
important new material.

\section{Strand Spaces With State}\label{sec:strand spaces}

The parameters to the strand space theory with state are a set
of messages (\msg), and a carried by relation
(${\sqsubseteq}\subseteq\msg\times\msg$).

The set of messages~{\msg} is often the carrier set of a message
algebra.  Intuitively, a message~$m_0$ is carried by~$m_1$
($m_0\sqsubseteq m_1$) if it is possible to extract~$m_0$ from~$m_1$.

In strand space theory, the \emph{trace} of a strand is a linearly
ordered sequence of events $e_0\Rightarrow\cdots\Rightarrow e_{n-1}$,
and an \emph{event} is a message transmission~$\outbnd m$, a
reception~$\inbnd m$, a state initialization~$\start m$, or a state
observation~$\obsv m$, where~$m\in\msg$, or a
\marginpar{!}
state transition~$\sync t$, where~$t\in\msg\times\msg$.  A
\emph{strand space}~$\ssp$ is a map from a set of strands to a set of
traces.  We choose the set of strands to be a prefix of the natural
numbers, so a strand space is finite sequence of traces.  The set of
strands of strand space {\ssp} is $\strands(\ssp)=\dom(\ssp)$.

A node names an event in a strand space.  The set of \emph{nodes} of
strand space $\ssp$ is $\{(z,i)\mid z\in\strands(\ssp), 0\leq i <
|\ssp(z)|\}$, and the event at a node is $\evt_\ssp(z,i)=\ssp(z)(i)$.
A node is a \emph{path node} in~$\ssp$ iff the event at the node is a
state initialization or a transition.  The set of nodes of~$\ssp$ is
$\nodes(\ssp)$ and the set of transition nodes of~$\ssp$ is $\nnodes(\ssp)$.

A message \emph{originates} in trace~$c$ at index~$i$ iff
\begin{enumerate}
\item $c(i)$ is the transmission~$\outbnd m$, it is carried by~$m$,
  and it is not carried by any event earlier in the trace, or
\item $c(i)$ is the initialization~$\start m$, it is carried by~$m$,
  \marginpar{!}
  and it is not carried by any event earlier in the trace, or
\item $c(i)$ is the transition event $\sync(m_0,m_1)$, the message is
  carried by $m_1$, and it is not carried by $m_0$ or any event
  earlier in the trace.
\end{enumerate}

A message~$m$ is \emph{non-originating} in strand space~$\ssp$,
written $\fn{non}(\ssp,m)$, if it originates at no node.  A
message~$m$ \emph{uniquely originates} in strand space~$\ssp$ at
node~$n$, written $\fn{uniq}(\ssp,m,n)$, if it originates at~$n$ and
nowhere else.

The model of execution is a bundle.  The triple
$\bun=(\ssp,\to,\leadsto)$ is a \emph{bundle} iff it defines a finite
directed acyclic graph, where the vertices are the nodes of $\ssp$,
and an edge represents communication~($\to$), state
passing~($\leadsto$), or strand succession~($\Rightarrow$) in~$\ssp$.

For communication, if $n_0\rightarrow n_1$, then there is a
message~$t$ such that $\evt_\ssp(n_0)=\outbnd m$
and $\evt_\ssp(n_1)=\inbnd m$.  For each reception node~$n_1$, there
is a unique transmission node~$n_0$ with $n_0\to n_1$.

For state passing, if $n_0\leadsto n_1$, then\marginpar{!}
\begin{enumerate}
\item $n_0$ and~$n_1$ are transition nodes, and there is a message~$m$
  such that $\evt_\ssp(n_0)=\sync(m_0,m)$ and
  $\evt_\ssp(n_1)=\sync(m,m_1)$, or
\item $n_0$ is an initialization node and~$n_1$ is a transition node,
  and there is a message~$m$ such that $\evt_\ssp(n_0)=\start m$ and
  $\evt_\ssp(n_1)=\sync(m,m_1)$, or
\item $n_0$ is a transition node and~$n_1$ is an observation node, and
  there is a message~$m$ such that $\evt_\ssp(n_0)=\sync(m_0,m)$ and
  $\evt_\ssp(n_1)=\obsv m$, or
\item $n_0$ is an initialization node and~$n_1$ is an observation
  node, and there is a message~$m$ such that $\evt_\ssp(n_0)=\start m$
  and $\evt_\ssp(n_1)=\obsv m$, or
\item $n_0$ is an observation node and~$n_1$ is a transition node, and
  there is a message~$m$ such that $\evt_\ssp(n_0)=\obsv m$ and
  $\evt_\ssp(n_1)=\sync(m,m_1)$.
\end{enumerate}
Additionally,
\begin{enumerate}
\item for all path or observation nodes~$n_0$, and transition
  nodes~$n_1$ and~$n_2$, $n_0\leadsto n_1$ and $n_0\leadsto n_2$ implies
  $n_1=n_2$, and 
\item for each transition or observation node~$n_1$, there exists a
  path node~$n_0$ such that $n_0\leadsto n_1$, and
\item for all path nodes~$n_0$, transition nodes~$n_1$, and
  observation nodes~$n_2$, $n_0\leadsto n_1$ and $n_0\leadsto n_2$
  implies $n_2\leadsto n_1$.
\end{enumerate}

Each acyclic graph has a transitive irreflexive relation~$\prec$ on
its vertices.  The relation specifies the causal ordering of nodes in
a bundle.  A transitive irreflexive binary relation is also called a
strict order.

For a bundle $\bun$, its associated strand space will be denoted
$\ssp_\bun$ unless the association is clear from the context.

With the definitions of origination and bundles given here, strand
spaces with state retains a key property of the original version
of strand spaces.

\begin{lem}\label{lem:carried originates}
  If message~$m$ is carried by $\evt_{\ssp_\bun}(n)$, then~$m$
  originates in~\bun.
\end{lem}
\begin{proof}
  By induction on the graph of {\bun} and a case analysis of the
  events at~$n$ that carry~$m$.
  \begin{enumerate}
  \item If $n$ is a transmission, then either $m$ originates at $n$ or~$m$
    is carried earlier in the strand by the definition of origination.
  \item If $n$ is a reception, then there is an earlier transmission node
    that carries~$m$.
  \item If $n$ is an observation, then there is an earlier path node
    that carries~$m$.
  \item If $n$ is a initialization, then either $m$ originates at $n$ or~$m$
    is carried earlier in the strand.
  \item If $\evt_{\ssp_\bun}(n)=\sync(m_0,m_1)$ and $m\sqsubseteq
    m_0$, then there is an earlier path node that carries~$m$.
  \item If $\evt_{\ssp_\bun}(n)=\sync(m_0,m_1)$ and $m\sqsubseteq m_1$,
    then $m\sqsubseteq m_0$, or~$m$ originates at~$n$, or~$m$ is carried
    earlier in the strand.
  \end{enumerate}
  \vspace{-4.4ex}
\end{proof}

In the remainder of this section, the theory of strand spaces used in
the proofs has been simplified.  In the full theory, origination
assumptions can be inherited from roles.  See~\cite{Ramsdell13} for
all the gory details.

When a bundle is a run of a protocol, the behavior of each strand is
constrained by a role.  Adversarial strands are constrained by roles
as are non-adversarial strands.  A \emph{protocol} is a set of roles,
and a \emph{role} is a set of traces.  A trace~$c$ is an
\emph{instance} of role~$r$ iff~$c$ is a prefix of some member of~$r$.
For protocol~$P$, bundle $\bun=(\ssp,\to,\leadsto)$ is a \emph{run of
  protocol}~$P$ iff there exists a role assignment $\ra\in
\strands(\ssp)\to P$ such that for all $z\in\strands(\ssp)$, $\ssp(z)$ is an
instance of~$\ra(z)$.

\section{State and Compatibility}

The parameters to the state theory are
\begin{enumerate}
\item a set of states (\sta),
\item a set of labels ($\lab\subseteq\msg$),
\item a set of initial states ($\init\subseteq\sta$),
\item a labeled state transition relation
  ($\tran\subseteq\sta\times\lab\times\sta$), and
\item an injective state encoding function ($f\in\sta\to\msg$).
\end{enumerate}

Let {\pth} be a finite sequence of states~\sta, and~{\labs} be a
finite sequence of labels~\lab.  The pair $\comp=(\pth,\labs)$ is a
\emph{computation} iff
\begin{enumerate}
\item $|\pth|=|\labs|+1$, and
\item $\pth(0)\in\init$, and
\item \(\all{i<|\labs|}(\pth(i),\labs(i),\pth(i+1))\in\tran\).
\end{enumerate}
A \emph{computation family} is a finite sequence of computations.  The
set of positions in family~{\family } is \(\pos(\family)= \{(i, j)\mid
i\in\dom(\family), (\pth,\labs)=\family(i), 0\leq j < |\labs|\}\).
For positions $(i_0,j_0)$ and $(i_1,j_1)$ in $\pos(\family)$,
$(i_0,j_0)\hookrightarrow(i_1,j_1)$ iff $i_0=i_1$ and $j_0=j_1+1$.

In the \bun,\family,$\phi$ model, each execution is a triple.
The~{\bun} refers to a bundle as described above, {\family} refers to
a computation family, and $\phi$ to a map from transition nodes to
positions in the family.

We define \bun,\family,$\phi$ to be a \emph{compatible triple} iff
\begin{enumerate}
\item $\phi$ is a bijection between transition nodes and 
  positions in~\family,
\item $\phi$ preserves the strict order~$\prec$, meaning
  that for all transition nodes~$n_0$ and~$n_1$, $n_0\prec n_1$ implies
  $\phi(n_0)\hookrightarrow\phi(n_1)$, and
\item $\phi$ preserves transitions, meaning that for 
$(i,j)=\phi(n)$ and $(\pth,\labs)=\family(i)$,
\marginpar{!}
$\evt(n)=\sync(f(\pth(j),f(\pth(j+1))$.
\item {\tran} has all observations, meaning for observation node~$n$
  with $\evt(n)=\obsv m$, there is some $q\in\sta$ and $\ell\in\lab$
  such that $(q,\ell,q)\in\tran$ and $m=f(q)$.
\end{enumerate}

One could add a requirement that the length of family~{\family} be the
same as the number of initialization nodes in~\bun, but this is
unnecessary.  Note that the structure of bundles and the definition of
a computation ensures that the state encoded by an initialization node
is an initial state.

\section{The \bun,\cbar\comp,$\phi$ Model}\label{sec:bcphi}

Strand spaces with states is a natural way of adding state to
strand spaces.  To bundles that contain message-passing edges, it adds
state-passing edges, and the rest follows.  However, the
state-passing model has a serious shortcoming.  State and message-passing
are intertwined in a way that makes it hard to reuse results on
slightly different problems.

The \bun,\cbar\comp,$\phi$ model~\cite{Guttman12} was designed to
address this shortcoming.  In this model, states are related using a
labeled transition system as in the \bun,\family,$\phi$ model.  In
fact, the \cbar{\comp} in the \bun,\cbar\comp,$\phi$ model is really a
computation family~\family.  Variables related to ones defined by
strand spaces with state will be barred with the exception of
computation families.  Thus, we disambiguate by writing
\cbar\bun,\cbar\comp,$\cbar\phi$ for the Guttman's model.

In the \cbar\bun,\cbar\comp,$\cbar\phi$ model, a state synchronization
event is a label.  A node is a \emph{neutral node} in~$\cbar\ssp$ iff
the event at the node is a state synchronization.  The set of neutral
nodes of~$\cbar\ssp$ is $\cnnodes(\cbar\ssp)$.  The bundle
$\cbar{\bun}=(\cbar\ssp,\to)$ omits the state-passing edges~$\leadsto$
from its associated graph along with the constraints associated with
state-passing.  In short, a bundle~\cbar{\bun} is just a strand space
bundle augmented with neutral nodes.  Function \cbar{\phi} is a map
from neutral nodes to positions in the family~\cbar\comp.

We define \cbar{\bun},\cbar\comp,\cbar{\phi} to be a
\emph{compatible triple} iff
\begin{enumerate}
\item \cbar{\phi} is a bijection between neutral nodes and 
  positions in~\cbar\comp,
\item \cbar{\phi} preserves the strict order~$\prec$, meaning
  that for all neutral nodes~$n_0$ and~$n_1$, $n_0\prec n_1$ implies
  $\cbar{\phi}(n_0)\hookrightarrow\cbar{\phi}(n_1)$, and
\item \cbar{\phi} preserves transitions, meaning that for
  $(i,j)=\cbar{\phi}(n)$ and $(\pth,\labs)=\cbar\comp(i)$,
  $\evt(n)=\neutral(\labs(j)).$
\end{enumerate}

\section{Relating Models}

\emph{This section needs help.  Unresolved issues follow.}

\begin{enumerate}
\item How does one reflect the definition of origination in the
  \bun,\family,$\phi$ model into the \cbar\bun,\cbar\comp,$\cbar\phi$
  via constrains on labels imposed by \tran?
\item What should be done about initialization nodes?
  \bun,\family,$\phi$ has them, but \cbar\bun,\cbar\comp,$\cbar\phi$
  does not.
\item What should be done about observation nodes?  They are not in
  the domain of $\phi$ in \bun,\family,$\phi$, but they are in
  the domain of $\cbar\phi$ in \cbar\bun,\cbar\comp,$\cbar\phi$.
\end{enumerate}

\section{Message Model}\label{sec:message model}

Typically, messages are modeled by elements of an order-sorted
algebra~\cite{GoguenMeseguer92}. An order-sorted algebra is a
generalization of a many-sorted algebra in which sorts may be
partially ordered.  The carrier sets associated with ordered sorts are
related by the subset relation.

\begin{figure}
$$\begin{array}{ll@{{}\typ{}}ll}
\mbox{Sorts:}&
\multicolumn{3}{l}{\mbox{$\srt{M}$, $\srt{A}$,
    $\srt{S}$, $\srt{D}$, $\srt{E}$}}\\
\mbox{Subsorts:}&
\multicolumn{3}{l}{\mbox{$\srt{A}<\srt{M}$, $\srt{S}<\srt{M}$,
    $\srt{D}<\srt{M}$, $\srt{E}<\srt{M}$}}\\
\mbox{Operations:}&(\cdot,\cdot)&\srt{M}\times\srt{M}\to\srt{M}& \mbox{Pairing}\\
&\enc{\cdot}{(\cdot)}&\srt{M}\times\srt{A}\to\srt{M}&\mbox{Asymmetric encryption}\\
&\enc{\cdot}{(\cdot)}&\srt{M}\times\srt{S}\to\srt{M}&\mbox{Symmetric encryption}\\
&\#&\srt{\srt{M}}\to\srt{M}& \mbox{Hashing}\\
&\invk{(\cdot)}&\srt{A}\to\srt{A}& \mbox{Asymmetric key inverse}\\
&\invk{(\cdot)}&\srt{S}\to\srt{S}& \mbox{Symmetric key inverse}\\
&\cn{a}_i,\cn{b}_i&\srt{A}& \mbox{Asymmetric key constants}\\
&\cn{s}_i&\srt{S}& \mbox{Symmetric key constants}\\
&\cn{d}_i&\srt{D}& \mbox{Data constants}\\
&\cn{e}_i&\srt{E}& \mbox{Text constants}\\
&\tg{i}&\srt{M}& \mbox{Tag constants}\\
\mbox{Equations:}&\multicolumn{2}{l}{\invk{\cn{a}_i}=\cn{b}_i\quad
\invk{\cn{b}_i}=\cn{a}_i}
&(i\in\nat)\\
&\multicolumn{2}{l}{\all{k\typ\srt{A}}\invk{(\invk{k})}=k}
&\all{k\typ\srt{S}}\invk{k}=k
\end{array}$$
\caption{Crypto Algebra Signature}\label{fig:algebra signature}
\end{figure}

Figure~\ref{fig:algebra signature} shows the signature of the algebra
used in examples in this paper.  Sort~$\srt{M}$ is the sort of all
messages.  Messages of sort~$\srt{A}$ (asymmetric keys),
sort~$\srt{S}$ (symmetric keys), sort~$\srt{D}$ (data), and
sort~$\srt{E}$ (text) are called \emph{atoms}.  Messages are atoms,
tag constants, or constructed using encryption $\enc{\cdot}{(\cdot)}$,
hashing $\#(\cdot)$, and pairing $(\cdot,\cdot)$, where the comma
operation is right associative and parentheses are omitted when the
context permits.

The algebra~{\alg} is the initial quotient term algebra over the
signature.  The canonical representative for each element in the
algebra is the term that contains no occurrences of the inverse
operation~$\invk{(\cdot)}$.  At times, we conflate a message with its
canonical representative.  The carrier set~$\alg_\srt{M}$ for sort
\srt{M} is what is used to instantiate {\msg} in strand spaces with state.
For sort~$S$ in the signature, we write $m\typ S$ for $m\in\alg_S$.
%% For skeleton formulas, $\bun,\alpha\models m\typ S$ iff
%% $\alpha(m)\in\alg_S$.  For strands and nodes, sorts~\srt{Z}
%% and~\srt{N} have been added with the property that $\bun,\alpha\models
%% z\typ\srt{Z}$ iff $\alpha(z)\in\strands(\ssp_\bun)$ and
%% $\bun,\alpha\models n\typ\srt{N}$ iff $\alpha(n)\in\nodes(\ssp_\bun)$.

A message~$m_0$ is \emph{carried by}~$m_1$, written $m_0\sqsubseteq
m_1$ iff~$m_0$ can be extracted from a reception of~$m_1$, assuming
plaintext is extractable from encryptions.  In other
words,~$\sqsubseteq$ is the smallest reflexive, transitive relation
such that $m_0\sqsubseteq m_0$, $m_0\sqsubseteq (m_0, m_1)$,
$m_1\sqsubseteq (m_0, m_1)$, and $m_0\sqsubseteq\enc{m_0}{m_1}$.

\begin{figure}
$$\begin{array}{r@{{}={}}l}
\fn{create}(m\typ\base)&\outbnd m\qquad\fn{tag}_i=\outbnd\tg{i}\\
\fn{pair}(m_0\typ\srt{M}, m_1\typ\srt{M})&
\inbnd m_0\Rightarrow\inbnd m_1\Rightarrow\outbnd (m_0,m_1)\\
\fn{sep}(m_0\typ\srt{M}, m_1\typ\srt{M})&
\inbnd (m_0, m_1)\Rightarrow\outbnd m_0\Rightarrow\outbnd m_1\\
\fn{enc}(m\typ\srt{M}, k\typ\srt{A}|\srt{S})&
\inbnd m\Rightarrow\inbnd k\Rightarrow\outbnd \enc{m}{k}\\
\fn{dec}(m\typ\srt{M}, k\typ\key)&
\inbnd \enc{m}{k}\Rightarrow\inbnd\invk{k}\Rightarrow\outbnd m\\
\fn{hash}(m\typ\srt{M})&
\inbnd m\Rightarrow\outbnd \#m
\end{array}$$
\caption{Adversary Traces}\label{fig:adversary}
\end{figure}

The roles that constrain adversarial behavior are defined by the
functions in Figure~\ref{fig:adversary}.  The role defined by the
function is all the traces that it generates.  For example, the role
associated with the function \fn{pair} is $\{\fn{pair}(m_0,m_1)\mid
m_0,m_1\typ\srt{M}\}$.  For the encryption related roles, $k\typ\key$
asserts that~$k$ is either a symmetric or asymmetric key.  For the
create role, $m\typ\base$ asserts that~$m$ is an atom.

An atom~$m$ is \emph{penetrator non-originating} in bundle~{\bun} if
there is no strand in~{\bun} with a trace that begins with $\outbnd
m$.  The assumption has the effect of prohibiting the use of the
create role for the atom by the adversary.

\section{Wrap-Decrypt Protocol}

The state in the Wrap-Decrypt Protocol is a device that creates,
stores, and shields symmetric keys.  The device offers two operations
using the keys it stores.  It can encrypt a key using a key in the
store, called wrapping, or it can decrypt a message using a key.  A
goal of this device is that all of its keys remain shielded within it.
A key could be leaked if a key is used to wrap itself, and then the
wrapped key is decrypted.

The device enforces its security policy by associating an attribute
with each of its keys.  A key has one of three attributes, \ainit,
\awrap, and \adecrypt.  A key is created with attribute \ainit,
wrapping is allowed when a key has attribute \awrap, and decrypting is
allowed when a key has attribute \adecrypt. 

Attributes can be changed with the set wrap and set decrypt operation.
The device policy is that set wrap succeeds as long as key's attribute
is not {\adecrypt} and set decryt succeeds as long as key's attribute
is not \awrap.  The remaining available operation of the device is
key making.

\begin{figure}
  \[
  \begin{array}{r@{}l}
    \fn{make}(q\typ\srt{S})={}&\start(\#k, \ainit)\Rightarrow\#k\\[1.2ex]
    \fn{setwrapi}(k\typ\srt{S})={}&\sync((\#k,\ainit),(\#k,\awrap))\\
    \fn{setwrapw}(k\typ\srt{S})={}&\sync((\#k,\awrap),(\#k,\awrap))\\[1.2ex]
    \fn{setdecrypti}(k\typ\srt{S})={}&\sync((\#k,\ainit),(\#k,\adecrypt))\\
    \fn{setdecryptw}(k\typ\srt{S})={}&\sync
    ((\#k,\adecrypt),(\#k,\adecrypt))\\[1.2ex]
    \fn{wrap}(k_0,k_1\typ\srt{S})={}&
    \inbnd\#k_0\Rightarrow\inbnd\#k_1\Rightarrow
    \obsv(\#k_1,\awrap)\Rightarrow\outbnd\enc{k_0}{k_1}\\[1.2ex]
    \fn{decrypt}(m\typ\srt{M},k\typ\srt{S})={}&
    \inbnd\enc{m}{k}\Rightarrow\inbnd\#k\Rightarrow
    \obsv(\#k,\adecrypt)\Rightarrow\outbnd m
  \end{array}
  \]
  \caption{Wrap-Decrypt Traces}\label{fig:wrap-decrypt traces}
\end{figure}

Let $A=\{\ainit,\awrap,\adecrypt\}$.  The set of states
$\sta=\alg_\srt{S}\times A$.  The state encoding function
$f(k,a)=(\#k,f_A(a))$, where $f_A$ maps each attribute to a distinct
tag.  We use the attribute symbol to name the tag, so
$f(k,\ainit)=(\#k,\ainit)$.  The initial states
$\init=\{(k,\ainit)\mid k\in\alg_\srt{S}\}$.

Figure~\ref{fig:wrap-decrypt traces} displays the Wrap-Decrypt
Protocol traces.  In the full version of strand
spaces~\cite{Ramsdell13}, origination assumptions can be inherited
from roles.  This feature is used in the Wrap-Decrypt Protocol.  Every
instantiation of the \fn{make} role adds the assumption that the hash
of the key uniquely originates at the first node of a \fn{make}
strand.  Additionally, the role adds the assumption that the key is
penetrator non-originating.

An unlabeled transition system compatible with the
Wrap-Decrypt Protocol traces follows.  For
$\cbar{\tran}\subseteq\sta\times\sta$,
\[((k_0,a_0),(k_1,a_1))\in\cbar{\tran}\mbox{ iff
}k_0=k_1\land(a_0=a_1\lor a_0=\ainit).\]

\emph{At this point, labels should be added, and then used to link
  this version of the protocol to one with neutral nodes.}

\section{Discussion}

The definition of a bundle {\bun} is motivated by the recent
implementation of state semantics in CPSA~3.  In particular, the
inclusion of state transition events and node orderings implied by
initialization, transition, and observation nodes is new.  In this
model, the treatment of observations is greatly simplified, and is not
part of the state component of the model.  The state component need
only focus on state transitions.  The details of state are abstracted
away by the encoding function~$f$.

This model of strand spaces with state is much easier to specify in
PVS\@. The contents of a transition event need not be in the
transition relation, so one does not need to use subsets of the
transition relation or the like.

\bibliography{secureprotocols}
\bibliographystyle{plain}

\end{document}
