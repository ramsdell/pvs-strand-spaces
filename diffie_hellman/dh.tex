\documentclass[12pt]{article}

\newif\ifbaseatoms
\baseatomsfalse % \baseatomstrue

\usepackage{graphics}
\usepackage[matrix,arrow,curve]{xy}
\usepackage{amssymb}
\usepackage{amsmath}
\usepackage{amsthm}
\usepackage{url}
%\usepackage{hyperref}\hypersetup{colorlinks=true,}

\newtheorem{thm}{Theorem}
\newtheorem{lem}{Lemma}
\newcommand{\cpsa}{\textsc{cpsa}}
\newcommand{\pvs}{\textsc{pvs}}
\newcommand{\acp}{\textsc{acp}}
\newcommand{\cn}[1]{\ensuremath{\operatorname{\mathsf{#1}}}}
\newcommand{\dom}[1]{\ensuremath{\operatorname{\mathbf{#1}}}}
\newcommand{\fn}[1]{\ensuremath{\operatorname{\mathit{#1}}}}
\newcommand{\srt}[1]{\ensuremath{\mathsf{#1}}}
\newcommand{\typ}{\mathbin:}
\newcommand{\sdom}{\fn{Dom}}
\newcommand{\sran}{\fn{Ran}}
\newcommand{\seq}[1]{\ensuremath{\langle#1\rangle}}
\newcommand{\enc}[2]{\ensuremath{\{\!|#1|\!\}_{#2}}}
\newcommand{\inv}[1]{{#1}^{-1}}
\newcommand{\inbnd}{\mathord -}
\newcommand{\outbnd}{\mathord +}
\newcommand{\nat}{\ensuremath{\mathbb{N}}}
\newcommand{\zed}{\ensuremath{\mathbb{Z}}}
\newcommand{\all}[1]{\forall#1\mathpunct.}
\newcommand{\some}[1]{\exists#1\mathpunct.}
%\newcommand{\pow}[1]{\wp(#1)}
\newcommand{\pow}[1]{\mathcal P(#1)}
\newcommand{\prefix}[2]{#1\mid#2}
\newcommand{\init}{\fn{init}}
\newcommand{\resp}{\fn{resp}}
\newcommand{\run}{\mathcal{R}}
\newcommand{\pt}{\fn{pt}}
\newcommand{\form}{\mathcal{K}}
\newcommand{\sent}{\mathcal{S}}
\newcommand{\lang}{\mathcal{L}}
\newcommand{\interp}{\mathcal{I}}
\newcommand{\alg}[1]{\ensuremath{\mathfrak#1}}
\newcommand{\tr}{\ensuremath{\mathfrak C}}
\newcommand{\rl}{\fn{rl}}
\newcommand{\skel}{\mathsf{k}}
\newcommand{\insta}{\mathsf{i}}
\newcommand{\nodes}{\fn{nodes}}
\newcommand{\evt}{\fn{evt}}
\newcommand{\role}{\mathsf{r}}
\newcommand{\orig}{\mathcal{O}}

\newcommand{\boot}{\cn{b}}
\newcommand{\extend}{\cn{ex}(\cn{d},\boot)}
\newcommand{\tran}{\ensuremath{\tau}}
\newcommand{\pth}{\ensuremath{\pi}}
\newcommand{\type}{\ensuremath{\mathfrak T}}
\newcommand{\up}{\mathord\uparrow}
\newcommand{\down}{\mathord\downarrow}
\newcommand{\mcb}{\mathbin{\bar{\sqsubseteq}}}

\newcommand{\cpsacopying}{\begingroup
  \renewcommand{\thefootnote}{}\footnotetext{{\copyright} 2015 The
    MITRE Corporation.  Permission to copy without fee all or part of
    this material is granted provided that the copies are not made or
    distributed for direct commercial advantage, this copyright notice
    and the title of the publication and its date appear, and notice
    in given that copying is by permission of The MITRE
    Corporation.

    Approved for Public Release; Distribution Unlimited. Case Number
    14-2229.}\endgroup}

\title{Diffie-Hellman Algebra}
\author{John D.\ Ramsdell}

\begin{document}
\maketitle
\cpsacopying

To analyze Diffie-Hellman protocols, an exponentiation operation is
added to the message algebra.  The Diffie-Hellman Signature is in
Figure~\ref{fig:dh signature}.  Messages of sort~\srt{G} are
exponents, and messages of sort~\srt{B} are exponentiations of
constant~\cn{g}.  Messages of sort~$\srt{A}$ (asymmetric keys),
sort~$\srt{S}$ (symmetric keys), sort~$\srt{D}$ (data), and
sort~\srt{E} (basis elements), are Diffie-Hellman algebra
\ifbaseatoms
\emph{atoms}, along with messages of the form~$\cn{g}^x$ when~$x$ is a
variable of sort~\srt{E}.
\else
\emph{atoms}.
\fi

\begin{figure}
$$\begin{array}{ll@{{}\typ{}}ll}
\mbox{Sorts:}&
\multicolumn{3}{l}{\mbox{$\top$, $\srt{D}$, $\srt{K}$, $\srt{A}$, $\srt{S}$,
    $\srt{B}$, $\srt{F}$, $\srt{G}$, $\srt{E}$}}\\
\mbox{Subsorts:}&
\multicolumn{3}{l}{\mbox{$\srt{D}<\top$, $\srt{K}<\top$,
    $\srt{A}<\srt{K}$, $\srt{S}<\srt{K}$, $\srt{B}<\srt{K}$,}}\\
&\multicolumn{3}{l}{\mbox{$\srt{G}<\top$,
    $\srt{E}<\srt{G}$}}\\
\ifbaseatoms
\else
\mbox{Atoms:}&
\multicolumn{3}{l}{\mbox{\srt{D}, \srt{A}, \srt{S}, \srt{E}}}\\
\fi
\mbox{Operations:}&(\cdot,\cdot)&\top\times\top\to\top& \mbox{Pairing}\\
&\enc{\cdot}{(\cdot)}&\top\times\srt{K}\to\top&\mbox{Encryption}\\
&\inv{(\cdot)}&\srt{K}\to\srt{K}& \mbox{Key inverse}\\
&\cn{d}_i&\srt{D}& \mbox{Data constants}\\
&\cn{a}_i,\cn{b}_i&\srt{A}& \mbox{Asymmetric key constants}\\
&\cn{s}_i&\srt{S}& \mbox{Symmetric key constants}\\
&\cn{g}^{(\cdot)}&\srt{G}\to\srt{B}& \mbox{Generated symmetric keys}\\
&(\cdot)^{(\cdot)}&\srt{B}\to\srt{G}\to\srt{B}&\mbox{Exponentiation}\\
&(\cdot\cdot)&\srt{G}\times\srt{G}\to\srt{G}&\mbox{Group operation}\\
&1&\srt{G}&\mbox{Group identity}\\
&\inv{(\cdot)}&\srt{G}\to\srt{G}&\mbox{Group inverse}\\
&\cn{e}_j&\srt{E}&\mbox{Basis elements ($j<|E|$)}\\
\mbox{Equations:}&\multicolumn{2}{l}{\inv{\cn{a}_i}=\cn{b}_i\quad
\inv{\cn{b}_i}=\cn{a}_i}
&\inv{\cn{s}_i}=\cn{s}_i\quad(i\in\nat)\\
&\multicolumn{2}{l}{\all{x,y\typ\srt{G}}
(\cn{g}^x)^y=\cn{g}^{xy}}&\all{h\typ\srt{B}}\inv{h}=h\\
&\multicolumn{2}{l}{\mbox{Group equations\ldots}}
\end{array}$$
\caption{Diffie-Hellman Algebra Signature}\label{fig:dh signature}
\end{figure}

The exponent is a free abelian group using multiplicative notation.
The group equations are: $(xy)z=x(yz)$ (associativity), $1x=x1=x$
(identity element), $\inv{x}x=x\inv{x}=1$ (inverse element), and
$xy=yx$ (commutativity).  The Diffie-Hellman message algebra \alg{A}
is the initial quotient term algebra over the signature.
The atoms are $\alg{B}\subset\alg{A}$.

Let~\alg{G} be the carrier set associated with sort~\srt{G}, the
abelian group.  Because~\alg{G} is free, there is a basis
$E\subseteq\alg{G}$ with the property that there is a unique way to
write every element of the group as a finite linear combination of
elements of the basis with integer coefficients.

Function $\epsilon\typ E\to\zed$ is a \emph{representation function}
if $\epsilon(x)\neq 0$ for only a finite number of elements in its
domain.  An abelian group~\alg{G} is \emph{free with basis}~$E$ if
$\all{x\in\alg{G}}\some{!\epsilon}x=\prod_{y\in E}y^{\epsilon(y)}$,
where~$\epsilon$ is a representation function.  Each member of the
basis~$\cn{e}_i\in E$ is in the Diffie-Hellman Signature with
sort~\srt{E}.

The \emph{carried by} relation $\sqsubseteq$ on messages is the
smallest reflexive, transitive relation such that $t_0\sqsubseteq
t_0$, $t_0\sqsubseteq t_1$ if~$t_0=t_1$, $t_0\sqsubseteq (t_0, t_1)$,
$t_1\sqsubseteq (t_0, t_1)$, and $t_0\sqsubseteq\enc{t_0}{t_1}$.

\begin{figure}
$$\begin{array}{r@{{}={}}l}
\fn{create}(t\in\alg{B})&\seq{\outbnd t}\\
\fn{build}(t\in\{1,\cn{g}^1\})&\seq{\outbnd t}\\
\fn{pair}(t_0\typ\top, t_1\typ\top)&
\seq{\inbnd t_0,\inbnd t_1,\outbnd (t_0,t_1)}\\
\fn{sep}(t_0\typ\top, t_1\typ\top)&
\seq{\inbnd (t_0, t_1),\outbnd t_0,\outbnd t_1}\\
\fn{enc}(t\typ\top, k\typ\srt{K})&
\seq{\inbnd t,\inbnd k,\outbnd \enc{t}{k}}\\
\fn{dec}(t\typ\top, k\typ\srt{K})&
\seq{\inbnd \enc{t}{k},\inbnd\inv{k},\outbnd t}\\
\fn{mul}(x\typ\srt{G}, y\typ\srt{G})&
\seq{\inbnd x,\inbnd y,\outbnd xy}\\
\fn{inv}(x\typ\srt{G})&
\seq{\inbnd x,\outbnd \inv{x}}\\
\fn{exp}(h\typ\srt{B}, x\typ\srt{G})&
\seq{\inbnd h,\inbnd x,\outbnd h^x}
\end{array}$$
\caption{Diffie-Hellman Adversary Traces}\label{seq:dh adversary}
\end{figure}

Figure~\ref{seq:dh adversary} displays the trace of each adversary
role.  The parameter of the \fn{create} role is restricted to atoms.
In fact, the defining characteristic of an atom is it is that which
the adversary can create out of thin air modulo origination
assumptions.

\iffalse
The discrete logarithm problem is to find~$x$ given $\cn{g}^x$.  The
computational Diffie-Hellman problem is to find~$\cn{g}^{xy}$
given~$\cn{g}^x$ and~$\cn{g}^y$.  (\emph{Why is this here?})
\fi

\end{document}
